\chapter*{Resumen} \label{chp:abstract}

\pvlibpy{} es una librería para Python de código abierto cuyo principal objetivo es permitir que los usuarios desarrollen su propio código para investigar y simular sistemas fotovoltaicos. Extender la oferta de funcionalidades es de interés a toda la comunidad que respalda la iniciativa. La finalidad de este Trabajo Fin de Grado es mejorar la librería mediante la contribución de modelos científicos y su respectiva documentación para que sean de utilidad, así como de aportar ejemplos que faciliten su uso. La colaboración con el proyecto se realiza a través de la plataforma \textit{GitHub} para facilitar la revisión e integración de los cambios propuestos por parte de la comunidad. Se proponen numerosos cambios, entre los que destacan la implementación de modelos sobre geometría solar, pérdidas por sombreado, pérdidas por irradiancia heterogénea y sobre la irradiación en agrivoltaica. Además de las propuestas aceptadas, se identifican y se realizan mejoras en todo el proyecto, desde la documentación hasta los tests, solución de errores, revisión de otras propuestas y solventar dudas de las personas usuarias.

\textbf{Palabras Clave}: fotovoltaica, código libre, pvlib-python, simulación, modelado

%%%%%%%%%%%%%%%%%%%%%%%%%%%%%%%%%%%%%%%%%%%%%%%%%%%%%%%%%%%%%%%%%%%%%%%%%%%%%%%%

\newpage

%%%%%%%%%%%%%%%%%%%%%%%%%%%%%%%%%%%%%%%%%%%%%%%%%%%%%%%%%%%%%%%%%%%%%%%%%%%%%%%%

\chapter*{Abstract}

\pvlibpy{} is an open-source Python library aimed at allowing users to develop their own scripts to investigate and simulate photovoltaic systems. Expanding available functionalities is of interest to the community backing the initiative. The purpose of this Bachelor's Dissertation is to improve the library by contributing scientific models and their respective documentation, as well as adding examples to demonstrate their use. Collaboration with the project is done through the \textit{GitHub} platform to ease the review and integration of the proposed changes by the community. Numerous proposals are made, of which the following ones can be highlighted: the implementation of models on solar geometry, shading losses, non-uniform irradiance losses, and effective irradiance in agrivoltaic systems. Additionally to the accepted proposals, the project has been improved in all aspects, from documentation to tests, bug fixes, review of other proposals, and helping users.

\textbf{Keywords}: photovoltaic, open source, pvlib-python, simulation, modelling
