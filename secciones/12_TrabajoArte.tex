\chapter{Trabajo relacionado y Estado del Arte} \label{chp:state-of-the-art}

<<Breve explicación, por secciones, de los contenidos de este capítulo>>

%%%%%%%%%%%%%%%%%%%%%%%%%%%%%%%%%%%%%%%%%%%%%%%%%%%%%%%%%%%%%%%%%%%%%%%%%%%%%%%%
%%%%%%%%%%%%%%%%%%%%%%%%%%%%%%%%%%%%%%%%%%%%%%%%%%%%%%%%%%%%%%%%%%%%%%%%%%%%%%%%

<<Explicación, en secciones, del estado del arte de las diferentes tecnologías, estudios, sistemas, etc., que trata este Trabajo de Fin de Grado, con referencias a la información, artículos, estudios, etc. que se tratan, como a continuación: \cite{stein_models_2009}>>

La importancia de simulaciones y análisis previo a la implantación de sistemas fotovoltaicos tanto para inversores, operaciones de financiación y diseñadores ha dado lugar a múltiples herramientas software como \textit{PVsyst}, \textit{SAM} y muchas más \cite{stein_models_2009}.

Dentro de este grupo de herramientas, surge la iniciativa de código abierto PVLIB-MatLab con origen en los laboratorios de \textit{Sandia National Laboratories, EEUU} hacia el año 2009 \cite{pvlib_history}. Posteriormente, en 2013\footnote{\url{https://web.archive.org/web/20240411190506/https://pvlib-python.readthedocs.io/en/stable/\#history-and-acknowledgement}}, inicia el desarrollo de la versión en Python, que es la que se trata en este trabajo \cite{pvlib_python}.

\section{La librería pvlib python} \label{sct:pvlib}

\subsection{Objetivos de la librería} \label{ssct:pvlib:objetivos}

\subsection{Funcionalidades} \label{ssct:pvlib:funcionalidades}

\subsection{Estructura de la librería} \label{ssct:pvlib:estructura}

\subsection{Herramientas de desarrollo del proyecto} \label{ssct:pvlib:herramientas}

\begin{lstlisting}[language=Python]
import pvlib

\end{lstlisting}
