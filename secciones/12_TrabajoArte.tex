\chapter{Trabajo relacionado y Estado del Arte} \label{chp:state-of-the-art}

<<Breve explicación, por secciones, de los contenidos de este capítulo>>

%%%%%%%%%%%%%%%%%%%%%%%%%%%%%%%%%%%%%%%%%%%%%%%%%%%%%%%%%%%%%%%%%%%%%%%%%%%%%%%%
%%%%%%%%%%%%%%%%%%%%%%%%%%%%%%%%%%%%%%%%%%%%%%%%%%%%%%%%%%%%%%%%%%%%%%%%%%%%%%%%

<<Explicación, en secciones, del estado del arte de las diferentes tecnologías, estudios, sistemas, etc., que trata este Trabajo de Fin de Grado, con referencias a la información, artículos, estudios, etc. que se tratan, como a continuación: \cite{stein_models_2009}>>

La importancia de simulaciones y análisis con el menor error posible de sistemas fotovoltaicos tanto para inversores, operaciones de financiación y diseñadores ha dado lugar a múltiples herramientas software como \textit{PVsyst}, \textit{SAM} y muchas más \cite{stein_models_2009}.

