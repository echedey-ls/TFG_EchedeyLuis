\chapter{Impacto del trabajo} \label{chp:impacto}

En este capítulo se analizará el valor que aporta el trabajo realizado en este TFG.

La sección \ref{sct:impacto:general} se centrará en el impacto general del trabajo, mientras que la sección \ref{sct:impacto:ods} analizará el impacto del trabajo en relación con los Objetivos de Desarrollo Sostenible (ODS). Por último, se hace una valoración de competencias adquiridas por el alumno en la sección \ref{sct:impacto:personal}.

%%%%%%%%%%%%%%%%%%%%%%%%%%%%%%%%%%%%%%%%%%%%%%%%%%%%%%%%%%%%%%%%%%%%%%%%%%%%%%%%
%%%%%%%%%%%%%%%%%%%%%%%%%%%%%%%%%%%%%%%%%%%%%%%%%%%%%%%%%%%%%%%%%%%%%%%%%%%%%%%%

\section{Impacto general} \label{sct:impacto:general}

Este Trabajo Fin de Grado permite visibilizar y extender el uso de algunos modelos científicos relacionados con la fotovoltaica. Además, se ha facilitado el mantenimiento de la librería \textit{pvlib-python} y se ha mejorado la documentación de la misma, contribuyendo así a la persistencia de este proyecto. Incluso sin haberse añadido algunas de las contribuciones propuestas, el trabajo realizado ha permitido reforzar líneas de trabajo que antes ni se consideraban.

Se espera que algunos de los modelos sean bastante útiles para la comunidad científica y técnica, en especial aquellos que consisten en modelos de pérdidas. Estos modelos permiten optimizar el diseño de plantas y mejorar el rendimiento económico de la instalación. Además, la mejora de la documentación de la librería \textit{pvlib-python} facilita su uso y extensión, lo que puede llevar a un aumento de la comunidad de usuarios y contribuidores.

Dentro del programa subvencionado \textit{Google Summer of Code}, se ha trabajado estrechamente con otros compañeros desarrolladores y se les ha ayudado a desenvolverse en el proyecto. Esto ha permitido que se hayan añadido nuevas funcionalidades a la librería y se haya mejorado la calidad del código.

Asimismo, este trabajo se alinea con los objetivos de la Universidad pública de promover la investigación, la transferencia y democratización libre del conocimiento.

%%%%%%%%%%%%%%%%%%%%%%%%%%%%%%%%%%%%%%%%%%%%%%%%%%%%%%%%%%%%%%%%%%%%%%%%%%%%%%%%
%%%%%%%%%%%%%%%%%%%%%%%%%%%%%%%%%%%%%%%%%%%%%%%%%%%%%%%%%%%%%%%%%%%%%%%%%%%%%%%%

\section{Objetivos de Desarrollo Sostenible} \label{sct:impacto:ods}

Con este trabajo se ha contribuido a la consecución de los Objetivos de Desarrollo Sostenible (ODS):

\begin{itemize}
  \item \textbf{ODS 3: Salud y bienestar}. Al potenciar el uso de energías renovables, se contribuye a la reducción de la contaminación y, por tanto, a la mejora de la salubridad del medio ambiente.
  \item \textbf{ODS 7: Energía asequible y no contaminante}. Debido a que se facilitan herramientas de diseño y análisis de instalaciones fotovoltaicas, se contribuye a la mejora de la eficiencia de las mismas y, por tanto, a la reducción del impacto de instalar paneles solares.
  \item \textbf{ODS 9: Industria, innovación e infraestructura}. Se ha trabajado en la mejora de la infraestructura de la librería \textit{pvlib-python}, lo que facilita la innovación y desarrollo en el sector de la energía fotovoltaica.
  \item \textbf{ODS 11: Ciudades y comunidades sostenibles}. Las mejoras facilitadas permiten implementar instalaciones fotovoltaicas de forma más fiable, lo que contribuye a su implantación.
  \item \textbf{ODS 13: Acción por el clima}. Este ODS sigue en la misma línea que los demás: como fuente de energía renovable, y en especial por ser la fotovoltaica, se reduce la emisión de gases de efecto invernadero y se contribuye a la lucha contra el cambio climático.
\end{itemize}

%%%%%%%%%%%%%%%%%%%%%%%%%%%%%%%%%%%%%%%%%%%%%%%%%%%%%%%%%%%%%%%%%%%%%%%%%%%%%%%%
%%%%%%%%%%%%%%%%%%%%%%%%%%%%%%%%%%%%%%%%%%%%%%%%%%%%%%%%%%%%%%%%%%%%%%%%%%%%%%%%

\section{Impacto académico en el autor} \label{sct:impacto:personal}

Este trabajo ha permitido al estudiante adquirir una serie de habilidades y conocimientos que le han sido y le continuarán siendo útiles en su carrera profesional. Además, le ha permitido conocer de primera mano cómo se trabaja en un proyecto de código abierto para poder aplicarlo en actuales proyectos de código abierto que mantiene.

Si bien puede destacar que la adquisición de conocimiento del lenguaje de Python no ha sido muy destacable, toda la parte de herramientas y la plataforma de GitHub ha sido muy enriquecedora:

\begin{itemize}
  \item Se ha aprendido a usar procedimientos de integración y desarrollo continuo.
  \item Se ha aprendido a desarrollar tests comprensivos y útiles.
  \item Se ha aprendido a escribir documentación de calidad.
  \item Se ha reforzado la capacidad de trabajo colaborativo en código.
  \item Se han mejorado las habilidades sociales y de comunicación a través de plataformas como GitHub.
  \item Se ha mejorado la destreza de comunicación en inglés, tanto oral como escrita.
\end{itemize}
