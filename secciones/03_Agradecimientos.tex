\chapter*{Agradecimientos} \label{chp:agrad}

Me gustaría agradecer a mis tutores César Domínguez Domínguez y Rubén Núñez Judez por darme la posibilidad de invertir mi capacidad y desarrollarme en un proyecto que se alinea con mis objetivos de autorrealización, así como por ofrecerme medios suficientes para formarme en el campo de lo fotovoltaico. Además, por darme consejos y orientación, como contarme sobre el sesgo que existe en la ciencia en torno a centrar los análisis en el hemisferio norte, que posteriormente ha tenido impacto una contribución de este proyecto.

A la UPM por ofrecer máquinas virtuales de Linux al alumnado, tanto para el desarrollo de este proyecto como para hacer pruebas en terceros proyectos.

A Nuria Martín Chivelet por explicarme detalladamente el funcionamiento de su modelo científico y ofrecerme continuar en esa misma línea de trabajo.

A Tomás García Aguado, responsable del departamento de pruebas de alta tensión del \textit{Laboratorio Central Oficial de Electrotecnia, LCOE}, por facilitar información sobre los datos que incluyen sus informes de transformadores y así ayudar en la toma de alguna decisión.

A todos los mantenedores de la librería \textit{pvlib-python} por sus revisiones en profundidad. En especial a Kevin Anderson y a Adam Jensen por ofrecerme y guiarme en obtener una beca bajo el programa \textit{Google Summer of Code}\footnote{\url{https://summerofcode.withgoogle.com/programs/2024/projects/fxPFQqZc}.}.

A todas las personas que públicamente han contribuido directa o indirectamente en crear ecosistemas de desarrollo de software libre y de código abierto, en especial a aquellos involucrados en la comunidad de Python y Visual Studio Code entre innumerables otros.

Finalmente, a Aurelio Acevedo Rodríguez por mostrarme la importancia de la sección de agradecimientos. En paz descanse.
