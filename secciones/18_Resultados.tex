\chapter{Resultados y conclusiones} \label{chp:resultados}

En este capítulo se resumen los resultados obtenidos en el desarrollo del Trabajo Fin de Grado, aquellos tanto satisfactorios como los que no, y se extraen conclusiones personales del estudiante. Además, se proponen posibles líneas de trabajo futuro, algunas que seguirán desarrollándose y otras que se podrían seguir a partir de este TFG.

%%%%%%%%%%%%%%%%%%%%%%%%%%%%%%%%%%%%%%%%%%%%%%%%%%%%%%%%%%%%%%%%%%%%%%%%%%%%%%%%
%%%%%%%%%%%%%%%%%%%%%%%%%%%%%%%%%%%%%%%%%%%%%%%%%%%%%%%%%%%%%%%%%%%%%%%%%%%%%%%%

\section{Resultados} \label{sct:resultados_resultados}

A pesar de que los ciclos de revisiones y correcciones toman un elevado tiempo de desarrollo, se han conseguido implementar varios modelos y funcionalidades nuevos en \textit{pvlib-python}. Además, se ha mejorado la documentación de la librería y se han corregido errores en la misma. Cabe destacar:

\begin{enumerate}
    \item Exponer el cálculo de la proyección del cenit solar sobre las coordenadas de un colector.
    \item Implementar un cálculo de la fracción de sombra unidimensional.
    \item La implementación de un modelo de pérdidas de potencia por sombreado, según número de diodos de bypass.
    \item Un modelo de fracción de irradiancia difusa fotosintetizable, en función de la irradiancia difusa global.
    \item Un modelo de pérdidas de potencia por heterogeneidad en la irradiancia incidente, de interés para módulos bifaciales.
    \item Dos funciones de conversión recíprocas de responsividad espectral y eficiencia cuántica externa, con posibilidad de normalizar la salida.
    \item Adición de una función general para obtener espectros estándares, que cuenta con el estándar ASTM G173-03, pero que será muy fácilmente extendible.
    \item Solventar 5 bugs y 2 características nuevas de las que han informado usuarios varios.
    \item Mejorar la documentación ya existente de forma amplia.
    \item Participar en detectar posibles contribuciones para primeros contribuyentes en el futuro.
    \item Aportar ideas y planteamientos que faciliten contribuir a la librería.
\end{enumerate}

Algunas cifras significativas, para la librería \textit{pvlib-python} son:

\begin{itemize}
    \item Se ha abierto 34 \textit{pull requests}\footnote{Consultar última información en \url{https://github.com/pvlib/pvlib-python/pulls?q=is\%3Apr+author\%3Aechedey-ls+}.}.
    \begin{itemize}
        \item De las cuales 23 se han aceptado.
        \item De las cuales 5 se han desestimado.
        \item De las cuales 6 se mantienen abiertas.
    \end{itemize}
    \item Se ha revisado 16 \textit{pull requests} de otros usuarios\footnote{Consultar última información en \url{https://github.com/pvlib/pvlib-python/pulls?q=is\%3Apr+reviewed-by\%3Aechedey-ls+-author\%3Aechedey-ls+}.}.
    \item Se ha abierto 19 \textit{issues}\footnote{Consultar última información en \url{https://github.com/pvlib/pvlib-python/issues?q=is\%3Aissue+author\%3Aechedey-ls+}.}.
    \begin{itemize}
        \item De las cuales se han cerrado 11.
    \end{itemize}
    \item Se ha participado en 26 \textit{issues} de otros usuarios\footnote{Consultar última información en \url{https://github.com/pvlib/pvlib-python/issues?q=is\%3Aissue+-author\%3Aechedey-ls+involves\%3Aechedey-ls+}.}.
    \begin{itemize}
        \item De las cuales se han cerrado 13.
    \end{itemize}
\end{itemize}

Adicionalmente se aprecia impacto en otros repositorios como:

\begin{itemize}
    \item \textit{pvlib/solarfactors}: un repositorio para el cálculo de factores de vista de sistemas bifaciales, en una escena 2D.
    \begin{itemize}
        \item Se ha informado de problemas de compatibilidad con la versión de Python 3.12 por el uso de la dependencia \texttt{shapely<2}\footnote{Véase \url{https://github.com/pvlib/solarfactors/issues/16}.}. Esta problemática es la anteriormente descrita en \ref{sct:desarrollo:soporte_tecnico}.
    \end{itemize}
    \item \textit{openpvtools/pv-foss-engagement}: una página con información sobre distintas librerías de Python empleadas en fotovoltaica.
    \begin{itemize}
        \item Se ha informado de la ausencia de \textit{solarfactors} en la lista de librerías\footnote{Véase \url{https://github.com/openpvtools/pv-foss-engagement/issues/8}.} y se han iniciado los cambios para mostrarlo en la web.
    \end{itemize}
\end{itemize}

%%%%%%%%%%%%%%%%%%%%%%%%%%%%%%%%%%%%%%%%%%%%%%%%%%%%%%%%%%%%%%%%%%%%%%%%%%%%%%%%
%%%%%%%%%%%%%%%%%%%%%%%%%%%%%%%%%%%%%%%%%%%%%%%%%%%%%%%%%%%%%%%%%%%%%%%%%%%%%%%%

\section{Conclusiones} \label{sct:resultados:conclusiones}

Con la realización de este Trabajo Fin de Grado, se promueven y mejoran proyectos de código abierto ya establecidos, lo que garantiza la usabilidad de las aportaciones aceptadas.

Se han mejorado múltiples aspectos de la librería \textit{pvlib-python}, desde errores menores en el código a la implementación de nuevos modelos y funcionalidades. Se ha mejorado la documentación, dotando al proyecto de una mayor usabilidad. El aporte que realiza este TFG es enriquece transversalmente a todos los efectos de este repositorio y se espera que sea de utilidad para la comunidad científica y técnica. Se espera que además se facilite el acceso de nuevos contribuyentes a la librería.

Desde un punto de vista de los logros, es destacable el impacto que este proyecto tiene iniciativas de código abierto. Por otro lado, hay dos dificultades principales que cabría destacar:

\begin{itemize}
    \item La exploración de artículos con modelos candidatos, a pesar de partir de una tanda inicial, encontrar otras posibles contribuciones adicionales supone un elevado esfuerzo y un resultado reducido.
    \item La planificación del trabajo, que responde a ciclos de trabajo extensos por el aprendizaje paralelo al desarrollo y, en especial, al lanzamiento de versiones, que es el factor principal que motiva la revisión de las propuestas.
\end{itemize}

Asimismo cabe destacar que se hay pequeños gestos que han hecho del proyecto un lugar más inclusivo, como identificar el sesgo hacia los ejemplos de fotovoltaica exclusivamente orientada al sur, el recibimiento a nuevos contribuyentes o la resolución de algunas dudas.

%%%%%%%%%%%%%%%%%%%%%%%%%%%%%%%%%%%%%%%%%%%%%%%%%%%%%%%%%%%%%%%%%%%%%%%%%%%%%%%%
%%%%%%%%%%%%%%%%%%%%%%%%%%%%%%%%%%%%%%%%%%%%%%%%%%%%%%%%%%%%%%%%%%%%%%%%%%%%%%%%

\section{Trabajo futuro} \label{sct:resultados:trabajofuturo}

Hay varias líneas de desarrollo que sería deseable continuar en el futuro, algunas de las cuales se han planteado en este TFG:

\begin{enumerate}
    \item Inicialmente se trabajará en contrastar la implementación del modelo descrito en \ref{sct:desarrollo:contribuciones_cientificas:heterogeneidad_irradiancia}, cuyos autores solicitan y facilitan información para contrastar la implementación con la suya.
    \item Se continuarán las propuestas ya planteadas, tanto dentro como fuera del programa subvencionado \textit{Google Summer of Code}. Y se priorizará dotar a la web de una mejor estética y usabilidad empleando las últimas versiones posibles.
    \item Además, posiblemente se intente continuar en la línea de solventar el problema de la dependencia \texttt{shapely<2} en \textit{solarfactors}, en función de si el equipo decide trabajar en ello.
    \item Por último, se ayudará a mejorar la documentación sobre cómo contribuir, y plantear una hoja de ruta para futuros contribuyentes que no hayan visto cómo programar en esta librería antes, pues se identifica que no todos los contribuyentes reciben formación sobre el uso de control de versiones en sus titulaciones.
\end{enumerate}
