\chapter{Introducción} \label{chp:intro}

La problemática medioambiental es un tema de actualidad que preocupa al futuro de la humanidad. Dentro de este marco, las energías renovables se presentan como una solución a la dependencia de los combustibles fósiles, que son los principales responsables de la emisión de gases de efecto invernadero; y por consiguiente, la energía solar fotovoltaica es una fuente muy prometedora, ya que es renovable y no contamina en su explotación directa.

Este Trabajo de Fin de Grado pretende potenciar la energía solar fotovoltaica, mejorando herramientas de simulación y diseño de instalaciones fotovoltaicas. Para ello, se han realizado múltiples contribuciones a un proyecto de código abierto llamado \textit{pvlib python}, que es una biblioteca escrita en Python para el análisis de sistemas fotovoltaicos.

%%%%%%%%%%%%%%%%%%%%%%%%%%%%%%%%%%%%%%%%%%%%%%%%%%%%%%%%%%%%%%%%%%%%%%%%%%%%%%%%
%%%%%%%%%%%%%%%%%%%%%%%%%%%%%%%%%%%%%%%%%%%%%%%%%%%%%%%%%%%%%%%%%%%%%%%%%%%%%%%%

\section{Motivación del proyecto} \label{sct:intro:motivacion}

El alumno declara que la motivación para la realización de este Trabajo de Fin de Grado se fundamenta en su interés por las energías renovables, el código libre y su ya experiencia previa en desarrollo de software de Python, también de acceso abierto.

Asimismo una de las principales inquietudes del alumno es aplicar sus conocimientos a generar nueva ciencia de forma accesible y contrastable, y que pueda ser utilizada por la comunidad científica y técnica de manera completamente democrática.

%%%%%%%%%%%%%%%%%%%%%%%%%%%%%%%%%%%%%%%%%%%%%%%%%%%%%%%%%%%%%%%%%%%%%%%%%%%%%%%%
%%%%%%%%%%%%%%%%%%%%%%%%%%%%%%%%%%%%%%%%%%%%%%%%%%%%%%%%%%%%%%%%%%%%%%%%%%%%%%%%

\section{Contexto del proyecto} \label{sct:intro:contexto}

Con el incrementado interés por las energías renovables y las facilidades del desarrollo software como caldo de cultivo, se ha propuesto al alumno la realización de este Trabajo de Fin de Grado, que se enmarca en el desarrollo de la biblioteca \textit{pvlib python}.

La propuesta parte de los tutores del alumno, que son miembros del grupo de investigación \textit{Instruments and Systems Integration} del \textit{Instituto de Energía Solar} de la propia Universidad Politécnica de Madrid. Asimismo, el proyecto \textit{pvlib python} es un proyecto de código abierto que se encuentra en desarrollo por otros investigadores de centros de investigación y universidades públicas de múltiples países, y algún miembro de empresas privadas.

El perfil de usuario de esta biblioteca presenta dos tipos principales:

\begin{itemize}

    \item[•] Investigadores que desean realizar simulaciones y estudios de sistemas fotovoltaicos

    \item[•] Ingenieros y técnicos que desean diseñar instalaciones fotovoltaicas de forma óptima
    
\end{itemize}

Por supuesto, tratándose de un proyecto de código abierto, cualquier persona puede utilizar la biblioteca, por lo que no se descarta la posibilidad de que otros perfiles de usuario puedan beneficiarse de las mejoras realizadas en este Trabajo de Fin de Grado.

%%%%%%%%%%%%%%%%%%%%%%%%%%%%%%%%%%%%%%%%%%%%%%%%%%%%%%%%%%%%%%%%%%%%%%%%%%%%%%%%
%%%%%%%%%%%%%%%%%%%%%%%%%%%%%%%%%%%%%%%%%%%%%%%%%%%%%%%%%%%%%%%%%%%%%%%%%%%%%%%%

\section{Objetivos} \label{sct:intro:objetivos}

La línea principal de este trabajo es la adición de nuevas funcionalidades a la biblioteca \textit{pvlib python}, que permitan mejorar la simulación, investigación y diseño de plantas fotovoltaicas. Para ello, se han establecido los siguientes objetivos:

\begin{itemize}

    \item[•] Contribuir modelos variados en propósito y utilidad

    \item[•] Añadir otras funcionalidades, que no siendo modelos, sean útiles para el usuario

    \item[•] Validar el funcionamiento mediante tests unitarios

    \item[•] Facilitar su uso con una documentación didáctica y concisa

    \item[•] Seguir los rigurosos estándares de calidad de un proyecto científico de código libre

\end{itemize}

Por otra parte, entre los objetivos secundarios destacan:

\begin{itemize}

    \item [•] Dar visibilidad a autores nacionales y de la Universidad Politécnica de Madrid
    
    \item [•] Potenciar proyectos de beneficio común desde la Universidad pública

\end{itemize}

%%%%%%%%%%%%%%%%%%%%%%%%%%%%%%%%%%%%%%%%%%%%%%%%%%%%%%%%%%%%%%%%%%%%%%%%%%%%%%%%
%%%%%%%%%%%%%%%%%%%%%%%%%%%%%%%%%%%%%%%%%%%%%%%%%%%%%%%%%%%%%%%%%%%%%%%%%%%%%%%%

\section{Estructura del Documento} \label{sct:intro_estructura}

<<Estructura de este Trabajo de Fin de Grado, explicando los contenidos de cada sección>>

%%%%%%%%%%%%%%%%%%%%%%%%%%%%%%%%%%%%%%%%%%%%%%%%%%%%%%%%%%%%%%%%%%%%%%%%%%%%%%%%
%%%%%%%%%%%%%%%%%%%%%%%%%%%%%%%%%%%%%%%%%%%%%%%%%%%%%%%%%%%%%%%%%%%%%%%%%%%%%%%%