\chapter{Introducción} \label{chp:intro}

El cambio climático es un tema de actualidad que plantea un reto social, económico y tecnológico. Dentro de este marco, las energías renovables se presentan como una solución tecnológica a la dependencia de los combustibles fósiles, que son los principales responsables de la emisión de gases de efecto invernadero. Una de las tantas fuentes de energía renovables más prometedoras es la solar \gls{fotovoltaica} \cite{Breyer_Bogdanov_Gulagi_Aghahosseini_Barbosa_Koskinen_Barasa_Caldera_Afanasyeva_Child_et_al_2017}, ya que es renovable y no contamina en su explotación directa.

Este Trabajo de Fin de Grado pretende potenciar la adquisición, investigación e implementación de la energía solar, mejorando herramientas de \gls{simulación} y diseño de instalaciones fotovoltaicas. Para ello, se han realizado múltiples contribuciones a un proyecto de código abierto llamado \pvlibpy{} \cite{Anderson_Hansen_Holmgren_Jensen_Mikofski_Driesse_2023, Stein_2012, Andrews_Stein_Hansen_Riley_2014, Holmgren_Andrews_Lorenzo_Stein_2015, Holmgren_Groenendyk_2016}, que es una biblioteca de herramientas escrita en \gls{Python} para el análisis de sistemas fotovoltaicos.

%%%%%%%%%%%%%%%%%%%%%%%%%%%%%%%%%%%%%%%%%%%%%%%%%%%%%%%%%%%%%%%%%%%%%%%%%%%%%%%%
%%%%%%%%%%%%%%%%%%%%%%%%%%%%%%%%%%%%%%%%%%%%%%%%%%%%%%%%%%%%%%%%%%%%%%%%%%%%%%%%

\section{Contexto del proyecto} \label{sct:intro:contexto}

Con el incremento del interés por las energías renovables y las facilidades del desarrollo \gls{software} como caldo de cultivo, se ha propuesto la realización de este Trabajo Fin de Grado, que consiste en aportar en el desarrollo de la biblioteca \pvlibpy{}.

El proyecto \pvlibpy{} es un proyecto de código abierto que se encuentra en desarrollo por otros investigadores de centros de investigación y universidades públicas de múltiples países, y algunos miembros de empresas privadas del mismo campo.

El perfil de las personas usuarias de esta biblioteca se puede clasificar entre 3 grupos principales:

\begin{itemize}

    \item Personal investigador que desean realizar simulaciones y estudios de sistemas fotovoltaicos.

    \item Personal técnico y de ingeniería que desean optimizar el diseño de instalaciones fotovoltaicas.

    \item Personal técnico y de ingeniería que quieren identificar fallos en este tipo de instalaciones, mediante la comparación de datos reales con simulaciones.

\end{itemize}

Por supuesto, tratándose de un proyecto de código abierto, cualquier persona puede utilizar la biblioteca, por lo que no se descarta la posibilidad de que otros perfiles de usuarios puedan beneficiarse de las mejoras realizadas en este Trabajo de Fin de Grado. Es por ello que esta iniciativa democratiza el acceso de personas más noveles en el campo de la energía solar fotovoltaica, que pudieran no tener acceso a herramientas comerciales.

%%%%%%%%%%%%%%%%%%%%%%%%%%%%%%%%%%%%%%%%%%%%%%%%%%%%%%%%%%%%%%%%%%%%%%%%%%%%%%%%
%%%%%%%%%%%%%%%%%%%%%%%%%%%%%%%%%%%%%%%%%%%%%%%%%%%%%%%%%%%%%%%%%%%%%%%%%%%%%%%%

\section{Motivación del proyecto} \label{sct:intro:motivacion}

La motivación para realizar este Trabajo de Fin de Grado es facilitar el desarrollo de avances en diseño e investigación para sistemas fotovoltaicos de forma libre, con las garantías de calidad y transparencia que ofrece el código abierto.

Con este TFG se pretende mejorar la fiabilidad de las simulaciones de sistemas fotovoltaicos.

%%%%%%%%%%%%%%%%%%%%%%%%%%%%%%%%%%%%%%%%%%%%%%%%%%%%%%%%%%%%%%%%%%%%%%%%%%%%%%%%
%%%%%%%%%%%%%%%%%%%%%%%%%%%%%%%%%%%%%%%%%%%%%%%%%%%%%%%%%%%%%%%%%%%%%%%%%%%%%%%%

\section{Objetivos} \label{sct:intro:objetivos}

La línea principal de este trabajo es la adición de nuevas funcionalidades a la biblioteca \pvlibpy{}, que permitan mejorar la simulación, investigación y diseño de plantas fotovoltaicas. Para ello, se han establecido los siguientes objetivos:

\begin{enumerate}

    \item Selección y estudio de modelos de operación de sistemas fotovoltaicos del estado del arte relevantes, que todavía no hayan sido incluidos en la librería de código abierto PVLIB Python.

    \item Implementación y publicación de los modelos siguiendo la metodología y estándares de documentación, test y desarrollo de ejemplos utilizada por la comunidad PVLIB Python.

    \item Contribución a la mejora de la implementación, documentación o ejemplos de uso de modelos ya existentes.

\end{enumerate}

Adicionalmente, se espera un impacto positivo, consecuencia natural y deseable del presente proyecto, que se describen en el \fullref{chp:impacto}.

%%%%%%%%%%%%%%%%%%%%%%%%%%%%%%%%%%%%%%%%%%%%%%%%%%%%%%%%%%%%%%%%%%%%%%%%%%%%%%%%
%%%%%%%%%%%%%%%%%%%%%%%%%%%%%%%%%%%%%%%%%%%%%%%%%%%%%%%%%%%%%%%%%%%%%%%%%%%%%%%%

\section{Estructura del documento} \label{sct:intro_estructura}

La estructura de este Trabajo de Fin de Grado pretende ser intuitiva y distribuida por bloques sobre temas similares, destacándose los siguientes:

\begin{itemize}

    \item \fullref{chp:state-of-the-art}: da a conocer el estado actual de la energía solar \gls{fotovoltaica} y algunas de las herramientas de \gls{simulación} utilizadas.

        \begin{itemize}
            \item \fullref{sct:pvlib}: expone las características de la biblioteca \pvlibpy{}.
        \end{itemize}

    \item \fullref{chp:desarrollo}: detalla el desarrollo de las contribuciones propuestas a la biblioteca, con factores tanto técnicos como humanos sobre el resultado.

    \item \fullref{chp:impacto}: presenta el impacto de las contribuciones realizadas tanto en esta biblioteca como en otras, y en la comunidad del proyecto.

    \item \fullref{chp:resultados}: resume y hace una valoración global de los resultados obtenidos.

        %% \item En el Capítulo \ref{chp:conclusiones}, se presentan las conclusiones y se proponen futuras líneas de trabajo TODO

\end{itemize}


%%%%%%%%%%%%%%%%%%%%%%%%%%%%%%%%%%%%%%%%%%%%%%%%%%%%%%%%%%%%%%%%%%%%%%%%%%%%%%%%
%%%%%%%%%%%%%%%%%%%%%%%%%%%%%%%%%%%%%%%%%%%%%%%%%%%%%%%%%%%%%%%%%%%%%%%%%%%%%%%%