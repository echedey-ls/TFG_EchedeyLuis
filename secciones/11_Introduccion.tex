\chapter{Introducción} \label{chp:intro}

La problemática medioambiental es un tema de actualidad que preocupa al futuro de la humanidad. Dentro de este marco, las energías renovables se presentan como una solución a la dependencia de los combustibles fósiles, que son los principales responsables de la emisión de gases de efecto invernadero. En este sentido, la energía solar es una fuente muy prometedora, ya que es renovable y no contamina en su explotación directa.

Este Trabajo de Fin de Grado pretende potenciar la energía solar fotovoltaica, mejorando herramientas de simulación y diseño de instalaciones fotovoltaicas. Para ello, se han realizado múltiples contribuciones a un proyecto de código abierto llamado \textit{pvlib python}, que es una biblioteca escrita en Python para el análisis de sistemas fotovoltaicos.

%%%%%%%%%%%%%%%%%%%%%%%%%%%%%%%%%%%%%%%%%%%%%%%%%%%%%%%%%%%%%%%%%%%%%%%%%%%%%%%%
%%%%%%%%%%%%%%%%%%%%%%%%%%%%%%%%%%%%%%%%%%%%%%%%%%%%%%%%%%%%%%%%%%%%%%%%%%%%%%%%

\section{Motivación del proyecto} \label{sct:intro:motivacion}

El alumno declara que la motivación para la realización de este Trabajo de Fin de Grado se fundamenta en su interés por las energías renovables, el código libre y su ya experiencia previa en desarrollo de software de Python, también de acceso abierto.

Asimismo una de las principales inquietudes del alumno es aplicar sus conocimientos a generar nueva ciencia de forma accesible y contrastable, y que pueda ser utilizada por la comunidad científica y técnica.

%%%%%%%%%%%%%%%%%%%%%%%%%%%%%%%%%%%%%%%%%%%%%%%%%%%%%%%%%%%%%%%%%%%%%%%%%%%%%%%%
%%%%%%%%%%%%%%%%%%%%%%%%%%%%%%%%%%%%%%%%%%%%%%%%%%%%%%%%%%%%%%%%%%%%%%%%%%%%%%%%

\section{Contexto del proyecto} \label{sct:intro:contexto}

Con el auge de las energías renovables y la democratización del desarrollo software como caldo de cultivo, se ha propuesto al alumno la realización de este Trabajo de Fin de Grado, que se enmarca en el desarrollo de la biblioteca \textit{pvlib python}.

La propuesta parte de los tutores del alumno, que son miembros del grupo de investigación \textit{Instruments and Systems Integration} del \textit{Instituto de Energía Solar} de la propia Universidad Politécnica de Madrid.

%%%%%%%%%%%%%%%%%%%%%%%%%%%%%%%%%%%%%%%%%%%%%%%%%%%%%%%%%%%%%%%%%%%%%%%%%%%%%%%%
%%%%%%%%%%%%%%%%%%%%%%%%%%%%%%%%%%%%%%%%%%%%%%%%%%%%%%%%%%%%%%%%%%%%%%%%%%%%%%%%

\section{Objetivos} \label{sct:intro:objetivos}

<<Breve explicación del objetivo principal de este Trabajo de Fin de Grado>>

<<Lista de objetivos específicos, con una breve explicación, en los que se divide el anterior objetivo principal que este Trabajo de Fin de Grado.>>

\begin{itemize}

    \item[•] Objetivo 1

    \item[•] Objetivo 2

    \item[•] \ldots

    \item[•] Objetivo N

\end{itemize}

%%%%%%%%%%%%%%%%%%%%%%%%%%%%%%%%%%%%%%%%%%%%%%%%%%%%%%%%%%%%%%%%%%%%%%%%%%%%%%%%
%%%%%%%%%%%%%%%%%%%%%%%%%%%%%%%%%%%%%%%%%%%%%%%%%%%%%%%%%%%%%%%%%%%%%%%%%%%%%%%%

\section{Estructura del Documento} \label{sct:intro_estructura}

<<Estructura de este Trabajo de Fin de Grado, explicando los contenidos de cada sección>>

%%%%%%%%%%%%%%%%%%%%%%%%%%%%%%%%%%%%%%%%%%%%%%%%%%%%%%%%%%%%%%%%%%%%%%%%%%%%%%%%
%%%%%%%%%%%%%%%%%%%%%%%%%%%%%%%%%%%%%%%%%%%%%%%%%%%%%%%%%%%%%%%%%%%%%%%%%%%%%%%%