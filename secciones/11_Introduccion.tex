\chapter{Introducción} \label{chp:intro}

El cambio climático es un tema de actualidad que plantea un reto social, económico y tecnológico. Dentro de este marco, las energías renovables se presentan como una solución tecnológica a la dependencia de los combustibles fósiles, que son los principales responsables de la emisión de gases de efecto invernadero. Una de las tantas fuentes de energía renovables más prometedoras es la solar fotovoltaica, ya que es renovable y no contamina en su explotación directa.

Este Trabajo de Fin de Grado pretende potenciar la adquisición, investigación e implementación de la energía solar, mejorando herramientas de simulación y diseño de instalaciones fotovoltaicas. Para ello, se han realizado múltiples contribuciones a un proyecto de código abierto llamado \textit{pvlib python}, que es una biblioteca de herramientas escrita en Python para el análisis de sistemas fotovoltaicos.

%%%%%%%%%%%%%%%%%%%%%%%%%%%%%%%%%%%%%%%%%%%%%%%%%%%%%%%%%%%%%%%%%%%%%%%%%%%%%%%%
%%%%%%%%%%%%%%%%%%%%%%%%%%%%%%%%%%%%%%%%%%%%%%%%%%%%%%%%%%%%%%%%%%%%%%%%%%%%%%%%

\section{Motivación del proyecto} \label{sct:intro:motivacion}

El alumno declara que la motivación para la realización de este Trabajo de Fin de Grado se fundamenta en su interés por las energías renovables, el código libre y su ya experiencia previa en desarrollo de software de Python, también de acceso abierto.

Asimismo una de las principales inquietudes del alumno es aplicar sus conocimientos a generar nueva ciencia de forma accesible y contrastable, y que pueda ser utilizada por la comunidad científica y técnica de manera completamente transparente.

%%%%%%%%%%%%%%%%%%%%%%%%%%%%%%%%%%%%%%%%%%%%%%%%%%%%%%%%%%%%%%%%%%%%%%%%%%%%%%%%
%%%%%%%%%%%%%%%%%%%%%%%%%%%%%%%%%%%%%%%%%%%%%%%%%%%%%%%%%%%%%%%%%%%%%%%%%%%%%%%%

\section{Contexto del proyecto} \label{sct:intro:contexto}

Con el incremento del interés por las energías renovables y las facilidades del desarrollo software como caldo de cultivo, se ha propuesto al alumno la realización de este Trabajo de Fin de Grado, que se enmarca en el desarrollo de la biblioteca \textit{pvlib python}.

La propuesta parte de los tutores del alumno, que son miembros del grupo de investigación \textit{Instruments and Systems Integration} del \textit{Instituto de Energía Solar} de la propia Universidad Politécnica de Madrid. Asimismo, el proyecto \textit{pvlib python} es un proyecto de código abierto que se encuentra en desarrollo por otros investigadores de centros de investigación y universidades públicas de múltiples países, y algunos miembros de empresas privadas del mismo campo.

El perfil de usuario de esta biblioteca presenta ciertos tipos principales:

\begin{itemize}

    \item[•] Investigadores que desean realizar simulaciones y estudios de sistemas fotovoltaicos.

    \item[•] Ingenieros y técnicos que desean optimizar el diseño de instalaciones fotovoltaicas.

    \item[•] Ingenieros y técnicos que quieren identificar faltas en este tipo de instalaciones, mediante la comparación de datos reales con simulaciones.
    
\end{itemize}

Por supuesto, tratándose de un proyecto de código abierto, cualquier persona puede utilizar la biblioteca, por lo que no se descarta la posibilidad de que otros perfiles de usuario puedan beneficiarse de las mejoras realizadas en este Trabajo de Fin de Grado. En este aspecto, sucede que esta iniciativa democratiza el acceso por parte de usuarios más noveles en el campo de la energía solar fotovoltaica, que no tienen acceso a herramientas comerciales.

%%%%%%%%%%%%%%%%%%%%%%%%%%%%%%%%%%%%%%%%%%%%%%%%%%%%%%%%%%%%%%%%%%%%%%%%%%%%%%%%
%%%%%%%%%%%%%%%%%%%%%%%%%%%%%%%%%%%%%%%%%%%%%%%%%%%%%%%%%%%%%%%%%%%%%%%%%%%%%%%%

\section{Objetivos} \label{sct:intro:objetivos}

La línea principal de este trabajo es la adición de nuevas funcionalidades a la biblioteca \textit{pvlib python}, que permitan mejorar la simulación, investigación y diseño de plantas fotovoltaicas. Para ello, se han establecido los siguientes objetivos:

\begin{itemize}

    \item[•] Contribuir modelos variados científicos en propósito y utilidad.

    \item[•] Añadir otras funcionalidades, que no siendo modelos, sean útiles para el usuario.

    \item[•] Validar el funcionamiento mediante tests unitarios.

    \item[•] Facilitar su uso con una documentación didáctica y concisa.

    \item[•] Seguir los rigurosos estándares de calidad de un proyecto científico de código libre.

    \item[•] Ayudar a la comunidad de usuarios de la biblioteca a través de la resolución de dudas y problemas.

    \item[•] Participar en la revisión de código de otros contribuyentes.

    \item[•] Crear asuntos y cuestiones que promuevan la mejora de la biblioteca, en especial para animar a otros contribuyentes a participar.

    \item[•] Ayudar a mantener la biblioteca actualizada para mejorar su ciclo de vida y arreglar fallos de forma preventiva.

\end{itemize}

Por otra parte, entre los objetivos secundarios destacan:

\begin{itemize}

    \item [•] Dar visibilidad a autores nacionales y de la Universidad Politécnica de Madrid.
    
    \item [•] Potenciar proyectos de beneficio común desde la Universidad pública.

\end{itemize}

Si bien no es el propósito específico de este trabajo tratar estos últimos objetivos, se considera que son una consecuencia natural y deseable del presente proyecto.

%%%%%%%%%%%%%%%%%%%%%%%%%%%%%%%%%%%%%%%%%%%%%%%%%%%%%%%%%%%%%%%%%%%%%%%%%%%%%%%%
%%%%%%%%%%%%%%%%%%%%%%%%%%%%%%%%%%%%%%%%%%%%%%%%%%%%%%%%%%%%%%%%%%%%%%%%%%%%%%%%

\section{Estructura del Documento} \label{sct:intro_estructura}

La estructura de este Trabajo de Fin de Grado pretende ser intuitiva y distribuida por bloques sobre temas similares. El lector podrá leer a continuación:

\begin{itemize}

    \item[•] En \fullref{chp:state-of-the-art}, se da a conocer el estado actual de la energía solar fotovoltaica y algunas de las herramientas de simulación utilizadas.

    \item[•] En \fullref{sct:pvlib}, se presentan las características de la biblioteca \textit{pvlib python}.

    \item[•] En \fullref{chp:desarrollo}, se detalla el desarrollo de las contribuciones propuestas a la biblioteca, con factores tanto técnicos como humanos sobre el resultado.

    \item[•] En \fullref{chp:resultados}, se resume el estado final o intermedio de las propuestas realizadas

    %% \item[•] En el Capítulo \ref{chp:conclusiones}, se presentan las conclusiones y se proponen futuras líneas de trabajo TODO
    
\end{itemize}


%%%%%%%%%%%%%%%%%%%%%%%%%%%%%%%%%%%%%%%%%%%%%%%%%%%%%%%%%%%%%%%%%%%%%%%%%%%%%%%%
%%%%%%%%%%%%%%%%%%%%%%%%%%%%%%%%%%%%%%%%%%%%%%%%%%%%%%%%%%%%%%%%%%%%%%%%%%%%%%%%